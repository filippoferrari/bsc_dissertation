\thispagestyle{plain}
\chapter*{Abstract}
This report presents a biologically inspired spiking neural network which can be used to recognise and track simple shapes moving in a video stream in real time. The network runs in biological real time on a 4 chips SpiNNaker board, a massively-parallel multi-core computing system, correctly identifying squares and diamonds shapes in synthetic videos. 

This problem could be easily solved using non-spiking neural networks, like convolutional neural networks, or traditional Computer Vision techniques. This project aims to show how this task can be solved using technologies able to simulate spiking neurons: the aforementioned SpiNNaker system and a Dynamic Vision Sensors software emulator.

This report provides a detailed description on how to design and implement such network using PyNN, a high level language based on Python used to create and simulate spiking neurons on SpiNNaker, and how neurobiology influenced the design of it. 

Correctly evaluating this spiking neural network poses several problems and the network behaves as expected only on synthetically generated videos and not on webcam recordings due to noisy recording and low spatial resolution.